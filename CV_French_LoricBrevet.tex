%----------------------------------------------------------------------------------------
%	DEFINITIONS
%----------------------------------------------------------------------------------------

\documentclass[10pt,a4paper]{moderncv}
\moderncvtheme[blue]{classic}                
\usepackage[utf8x]{inputenc}
\usepackage[scale=0.85]{geometry}
\usepackage[french,english]{babel}

%----------------------------------------------------------------------------------------
%	INFORMATIONS PERSONNELLES
%----------------------------------------------------------------------------------------

\firstname{Loric}
\familyname{Brevet}
\title{Étudiant en informatique}
\address{18 rue Léon Fabre}{69100 Villeurbanne France}
\mobile{+33 6 74 67 55 53}
% \phone{}
% \fax{}
\email{loric.brevet@gmail.com}
% \homepage{lien}{libelle}
\extrainfo{8 juin 1993 (20 ans)\\Permis B}
\photo[70pt][0.4pt]{pictures/picture}
\quote{Recherche d'un stage en conception et développement}

%----------------------------------------------------------------------------------------
%	DEBUT DU DOCUMENT
%----------------------------------------------------------------------------------------

\begin{document}
\maketitle
\pagestyle{empty}

%----------------------------------------------------------------------------------------
%	SECTION
%----------------------------------------------------------------------------------------

\section{Diplômes et Études}

\cventry
	{2012\\à Aujourd'hui}
	{Cycle Ingénieur en Informatique}
	{INSA de Lyon}
	{Villeurbanne}
	{}
	{Actuellement en 4\up{ème} année}

\cventry
	{2010 à 2012}
	{D.U.T Informatique}
	{Université Lyon 1}
	{antenne de Bourg-en-Bresse}
	{}
	{Spécialité mathématiques renforcées}

\cventry
	{2010}
	{B.A.C Scientifique}
	{Lycée Joseph Marie Carriat}
	{Bourg-en-Bresse}
	{}
	{Option sciences de l'ingénieur}

%----------------------------------------------------------------------------------------
%	SECTION
%----------------------------------------------------------------------------------------

\section{Experiences}

\cventry
	{Janvier 2014}
	{Application réseau autour de la maison connectée}
	{10 semaines}
	{}
	{Projet cycle ingénieur en groupe}
	{Développement d'un serveur en Python qui analyse les trames de capteurs et envoie des commandes à des actionneurs. Mise en place d'un client web afin de configurer et lire les données du serveur.}

\cventry
	{Juin 2013}
	{Stage développement web}
	{3 mois}
	{CMRE Logiciel}
	{Ceyzériat France}
	{Refonte d'un utilitaire de pose de congés desktop en application Web avec Java et le Framework JSF.}

\cventry
	{Hiver 2012}
	{Robot Autonomee}
	{}
	{}
	{Projet cycle ingénieur en groupe}
	{Construction d'un robot avec une carte Arduino et un RaspBerry PI. Développement client/serveur afin de le localiser et de le contrôler à distance depuis une tablette Android.}

\cventry
	{Avril 2012}
	{Stage développement web}
	{3 mois}
	{Aberdeen Ecosse}
	{rattaché à Robert Gordon University}
	{Création et développement d'un site internet avec PHP/CSS/HTML/JavaScript pour une association.}

\cventry
	{2010 à 2012}
	{Projets universitaires}
	{}
	{}
	{}
	{
		\begin{itemize}
			\item Chat IRC en Python
			\item Logiciel de comptabilité et jeu d'échecs en Java
			\item Mario like en C avec SFML
		\end{itemize}
	}

%----------------------------------------------------------------------------------------
%	SECTION
%----------------------------------------------------------------------------------------

\section{Compétences}

\subsection{Informatique}

\cvcomputer
	{Programmation}
	{C, C++, Java, Python, .NET, Bash, XML}
	{Conception}
	{UML, Merise}

\cvcomputer
	{Logiciels}
	{Word, Excel, Powerpoint, Photoshop, Visual Studio, Eclipse, Netbeans, IntelliJ}
	{Web}
	{XHTML/CSS, PHP, Javascript, Java Web}

\cvcomputer
	{Base de données}
	{Oracle, PostgreSQL, Access, MySQL}
	{Autre}
	{Notions de \LaTeX et Git. Compétences en architecture système et réseaux. Connaissance des OS à base GNU/Linux.}

\subsection{Langues}

\cvlanguage
	{Anglais}
	{bonnes notions \textnormal{\textit{(lu, parlé, écrit)}}}
	{}

\cvlanguage
	{Espagnol}
	{notions de base}
	{}

\cvlanguage
	{Allemand}
	{débutant}
	{}

%----------------------------------------------------------------------------------------
%	SECTION
%----------------------------------------------------------------------------------------

\section{Centres d'intérêt}

\cvline
	{Loisirs}
	{Football (niveau départemental), percussionniste dans un orchestre d'harmonie, guitare.}

\cvline
	{Associatif}
	{Responsable d'une association locale de gestion d'événements.}

%----------------------------------------------------------------------------------------
%	FIN DU DOCUMENT
%----------------------------------------------------------------------------------------

\end{document}